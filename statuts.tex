\documentclass{article}
\usepackage[utf8]{inputenc}
\usepackage[francais]{babel}
\usepackage[T1]{fontenc}
\usepackage{textcomp}
\usepackage{lmodern}
\usepackage{graphicx}
\usepackage{url}
\usepackage{xspace}

\title{Dead Pixels Society}
\author{Statuts de l'association}
\date{}

\renewcommand{\theenumi}{\alph{enumi}}
\renewcommand{\labelenumi}{\theenumi)}

\newcounter{cptrArt}
\newcommand{\Art}[1]{\refstepcounter{cptrArt}\subsection*{Article~\thecptrArt\xspace~-- #1}}


\begin{document}

\maketitle

\Art{Nom}

Il est fondé entre les adhérents aux présents statuts une association régie par la loi du 1\ier juillet 1901 et le décret du 16 août 1901, ayant pour titre:

\begin{center}
«Dead Pixels Society»
\end{center}


\Art{Objet}
\label{art:objet}

Cette association a pour objet la production et le développement de jeux vidéo; l'organisation d'événements autour du développement de jeux vidéo de type «gamejam»; la mise en place et la maintenance de serveurs de jeu pour les jeux produits par l'association; la vente des jeux produits par l'association ou de services de jeux produits par l'association.

L'association encourage la mise à disposition des jeux produits sous licence libre, ainsi que l'utilisation de resources multimédias ou informatiques sous licence libre.


\Art{Siège social}

Le siège social est fixé au Département d'Informatique de l'UFR Sciences et Technique, 16 route de Gray 25030 Besançon.

Il pourra être transféré par simple décision du bureau; la ratification par l'assemblée générale sera nécessaire.


\Art{Durée}

La durée de l'association est illimitée.


\Art{Composition}

L'association se compose de:

\begin{enumerate}
\item
  membres fondateurs;
\item
  membres actifs;
\item
  membres bienfaiteurs.
\end{enumerate}


\Art{Admission}

L'association est ouverte à tous, sans condition ni distinction.


\Art{Membres}

Sont membres fondateurs les membres qui ont participé à la fondation de l'association. Il s'agit de Julien \textsc{Bernard} et Frédéric \textsc{Dadeau}. Ils ont un droit de vote à l'assemblée générale.

Sont membres actifs les personnes physiques qui versent annuellement une cotisation dont le montant est fixé par l'assemblée générale et inscrit dans le règlement intérieur. Ils participent aux activités de l'association telles que définies à l'article~\ref{art:objet}. Ils ont un droit de vote à l'assemblée générale.

Sont membres bienfaiteurs les personnes physiques ou morales qui adressent des dons à l'association. Ils n'ont pas de droit de vote à l'assemblée générale.


\Art{Radiation}

La qualité de membre se perd par:

\begin{enumerate}
\item
  la démission;
\item
  le décès;
\item
  la radiation prononcée par le bureau pour non-paiement de la cotisation ou pour motif grave, l'intéressé ayant été invité par lettre recommandée à fournir des explications devant le bureau et/ou par écrit.
\end{enumerate}


\Art{Ressources}

Les ressources de l'association comprennent:

\begin{enumerate}
\item
  le montant des dons et des cotisations;
\item
  les subventions de l'État, des régions, des départements, des communes ou des établissements publics;
\item
  les revenus tirés de la vente des jeux produits par l'association ou par les services de jeux produits par l'association;
\item
  toutes les ressources autorisées par les lois et règlements en vigueur.
\end{enumerate}


\Art{Assemblée générale ordinaire}

L'assemblée générale ordinaire comprend tous les membres de l'association à quelque titre qu'ils soient, à jour de leur cotisation et faisant partie de l’association depuis au moins trois mois. Ceux-ci peuvent se faire représenter par un autre membre de l’association faisant partie de l’assemblée générale. Nul ne peut être titulaire de plus d'un mandat.

Elle se réunit chaque année au mois de décembre. Quinze jours au moins avant la date fixée, les membres de l'association sont convoqués par les soins du secrétaire. L'ordre du jour figure sur les convocations.

Le président, assisté des membres du bureau, préside l'assemblée et expose la situation morale ou l’activité de l'association.
Le trésorier rend compte de sa gestion et soumet les comptes annuels à l'approbation de l'assemblée.
L’assemblée générale fixe le montant des cotisations annuelles à verser par les différentes catégories de membres.
Ne peuvent être abordés que les points inscrits à l'ordre du jour.

Pour délibérer valablement, la présence de la moitié des membres ayant voix délibérative est exigée. Les décisions sont prises à la majorité des voix des membres présents ou représentés. Si le quorum n’est pas réuni, une seconde assemblée se tiendra dans le mois suivant et pourra délibérer valablement quel que soit le nombre de membres présents ou représentés. Les décisions des assemblées générales s’imposent à tous les membres, y compris absents ou représentés.

Il est procédé, après épuisement de l'ordre du jour, au renouvellement des membres sortants du bureau.


\Art{Assemblée générale extraordinaire}
\label{art:assemblee_generale_extraordinaire}

Si besoin est, ou sur la demande de la moitié plus un des membres inscrits, le président peut convoquer une assemblée générale extraordinaire, suivant les modalités prévues aux présents statuts et uniquement pour modification des statuts ou la dissolution ou pour des actes portant sur des immeubles.

Les modalités de convocation sont les mêmes que pour l’assemblée générale ordinaire. Les délibérations sont prises à la majorité des deux tiers des membres présents.


\Art{Bureau}

L'association est dirigée par un bureau de cinq membres. Les deux membres fondateurs sont membres de droit du bureau. Les trois autres membres du bureau sont élus chaque année par l'assemblée générale parmi les membres actifs. Les membres élus sont rééligibles. Le bureau désigne en son sein:

\begin{enumerate}
\item
  un président;
\item
  un secrétaire;
\item
  un trésorier.
\end{enumerate}

En cas de vacances, et si besoin est, le bureau pourvoit provisoirement au remplacement de ses membres. Il est procédé à leur remplacement définitif par la plus prochaine assemblée générale. Les pouvoirs des membres ainsi élus prennent fin à l'expiration du mandat des membres remplacés.

Le bureau se réunit chaque fois que nécessaire, sur convocation du président, ou à la demande des deux cinquièmes de ses membres. Les décisions sont prises à la majorité des voix; en cas de partage, la voix du président est prépondérante.

Tout membre du bureau qui, sans excuse, n'aura pas assisté à trois réunions consécutives sera considéré comme démissionnaire.


\Art{Indemnités}

Toutes les fonctions, y compris celles des membres du bureau, sont gratuites et bénévoles. Seuls les frais occasionnés par l’accomplissement de leur mandat sont remboursés sur justificatifs. Le rapport financier présenté à l’assemblée générale ordinaire présente, par bénéficiaire, les remboursements de frais de mission, de déplacement ou de représentation.


\Art{Règlement intérieur}

Un règlement intérieur peut être établi par le bureau, qui le fait alors approuver par l'assemblée générale.

Ce règlement éventuel est destiné à fixer les divers points non prévus par les présents statuts, notamment ceux qui ont trait à l'administration interne de l'association et sur la représentation des membres empêchés d’assister à l’assemblée générale. Il ne pourra comprendre aucune disposition contraire aux statuts.

\Art{Dissolution}

En cas de dissolution prononcée selon les modalités prévues à l’article~\ref{art:assemblee_generale_extraordinaire}, un ou plusieurs liquidateurs sont nommés, et l'actif, s'il y a lieu, est dévolu conformément aux décisions de l’assemblée générale extraordinaire qui statue sur la dissolution.


\bigskip


\begin{flushright}
Fait à \textsc{Besançon}, le 27 avril 2015
\end{flushright}

\begin{center}
Julien \textsc{Bernard} \hspace{5em} Frédéric \textsc{Dadeau}
\end{center}


\end{document}
