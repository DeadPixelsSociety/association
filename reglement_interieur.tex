\documentclass{article}
\usepackage[utf8]{inputenc}
\usepackage[francais]{babel}
\usepackage[T1]{fontenc}
\usepackage{textcomp}
\usepackage{lmodern}
\usepackage{graphicx}
\usepackage{url}
\usepackage{xspace}

\title{Dead Pixels Society}
\author{Règlement intérieur de l'association}
\date{}

\renewcommand{\theenumi}{\alph{enumi}}
\renewcommand{\labelenumi}{\theenumi)}

\newcounter{cptrArt}
\newcommand{\Art}[1]{\refstepcounter{cptrArt}\subsection*{Article~\thecptrArt\xspace~-- #1}}

\newcounter{cptrAlinea}[cptrArt]
\newcommand{\Alinea}{\stepcounter{cptrAlinea}\paragraph{\thecptrAlinea.}}

\begin{document}

\maketitle

\Art{Cotisation}

Le montant de la cotisation annuelle pour les membres actifs est de 2€ (deux euros).


\Art{Démission, exclusion, décès d'un membre}

\Alinea

La démission doit être adressée au président du bureau par lettre ou par courrier électronique. Elle n’a pas à être motivée par le membre démissionnaire.

\Alinea

Comme indiqué à l’article~8 des statuts, l’exclusion d’un membre peut être prononcée par le bureau, pour motif grave. Sont notamment réputés constituer des motifs graves:
\begin{itemize}
\item
  la non-participation aux activités de l’association;
\item
  une condamnation pénale pour crime et délit;
\item
  toute action de nature à porter préjudice, directement ou indirectement, aux activités de l’association ou à sa réputation.
\end{itemize}
En tout état de cause, l’intéressé doit être mis en mesure de présenter sa défense, préalablement à la décision d’exclusion. La décision d’exclusion est adoptée par le bureau statuant à la majorité des trois cinquième des membres présents.

\Alinea

En cas de décès d’un membre, les héritiers ou les légataires ne peuvent prétendre à un quelconque maintien dans l’association.

\Art{Modalités de vote à l'assemblée générale}


\Alinea

Le vote se tient à bulletin secret quand il concerne des personnes, en particulier pour l'élection du bureau. Il se tient à main levée dans les autres cas.

\Alinea

Comme indiqué à l’article~10 des statuts, si un membre de l’association ne peut assister personnellement à une assemblée, il peut s’y faire représenter par un mandataire dans les conditions indiquées audit article. Le secrétaire fait parvenir avec la convocation un formulaire à retourner au moins deux jours avant l'assemblée générale par celui qui veut se faire représenter.

\Alinea

Le bureau est élu par scrutin de listes à deux tours. Les deux listes arrivées en tête au premier tour sont qualifiées pour le second tour. La liste arrivée en tête au second tour est élue. S'il y a deux listes ou moins qui sont candidates, un seul tour est organisé.


\Art{Indemnités de remboursement}

Seuls les membres du bureau peuvent prétendre au remboursement des fais engagés dans le cadre de leurs fonctions et sur justifications.


\Art{Modification du règlement intérieur}

Le présent règlement intérieur pourra être modifié par le bureau. Il entrera en vigueur une fois que l'assemblée générale l'aura approuvée à la majorité des votes exprimés.


\bigskip

\begin{flushright}
Fait à \textsc{Besançon}, le 27 avril 2015
\end{flushright}

\begin{center}
Julien \textsc{Bernard} \hspace{5em} Frédéric \textsc{Dadeau}
\end{center}


\end{document}
